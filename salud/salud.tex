
Antes de empezar con el tema de salud, me gustaría aclarar que
%lo siguiente es sacado de
para desarrollar este aspecto me he basado en
la fuente primaria del 
presente trabajo, “La Alexiada”, del libro XV, el último de la obra. 
%En el que se narra
% Segunda Revisión
En este se relatan
las obras que había mandado hacer el emperador, 
según lo narra su hija, para atender las necesidades de
los huérfanos y demás gente afectada 
por las continuas guerras que estaba teniendo el imperio.

Este es el más corto de los capítulos, ya que en la fuente
que estaré utilizando, mencionada abajo, no son muchas
páginas ni mucho contenido que hable al respecto.

En el libro XV, los párrafos del 4 al 8 están dedicados a 
describir un hospital que el emperador había mandado a construir, 
siendo el único fragmento destinado a la salud dentro del libro 
(sin contar a la sanidad de campaña (militar) que se describe 
en el mismo libro XV, sección VII, párrafos 1 y 2).

En dicha cita, la autora narra la última campaña militar
del emperador, dónde describe el trato con `civiles' y
enfermos que tranportaban. Una vez llega a la ciudad,
se describe la obra de salud y educación.
Dónde, la parte de educación está en el capítulo ononimo,
mientras que prosigo con la descripción sobre salud.

\subsection{Salud de campaña}
En la Alexiada, se relata cómo marchaba el ejército, en la
última campaña militar del emperador, y hacía atender
a la gente que llevaba:

\textit{``Cuando llegó a Filomelio, tras liberar por doquier
	a quienes estaban bajo el yugo bárbaro [...], introdujo
	en medio de la formación a cautivos, mujeres, niños y
	todo el botín y se puso en el camino de vuelta apaciblemente
	con un movimiento lento y semejante al de las hormigas.
	''}

\textit{``
	Como había muchas mujeres embarazadas y otras muchas sufrían
	enfermedades, cuando a una mujer estaba a punto de dar a luz,
	tocaba la trompeta a una señal del soberano, todos se detenían
	enseguida y la formación entera se quedaba quieta en el mismo
	lugar. Nada más enterarse de que el parto había concluido,
	mandaba a dar otro toque, que no era de los habituales,
	comunicando la puesta en marcha y animando a todos a caminar.
	Si alguien se estaba muriendo, de nuevo sucedía lo mismo y el
	soberano se presentaba en el sitio donde yacía el moribundo,
	llamaba a los sacerdotes para que cantaran los himnos postreros
	y para que le dieran los sacramentos al agonizante y, una vez se
	habían celebrado las honras fúnebres a los difuntos de acuerdo
	con las normas sagradas, hasta que el muerto no estuviera enterrado,
	no permitía que se moviera lo más mínimo la formación. A la hora de comer,
	hacía llamar a todas las mujeres y hombres que estuvieran agotados
	por las enfermedades y la vejez, les ofrecía lo mejor de su comida
	y ordenaba a sus comensales que hicieran lo mismo.
	''}\footnote{\cite[pp.~604--605]{alexiadaXV}}

Quise poner este breve relato ya que cuenta el trato con los enfermos
y `civiles' que transportaba con su ejército.

\subsection{Salud en Constantinopla}

En cuanto a la salud en Constantinopla, cuando el emperador llega
a la ciudad, manda hacer un orfanato para los hijos de los soldados
caídos, y se reponsabiliza por la educación (que está en el capítulo
sobre educación en la parte sobre el emperador).

Y, junto con el orfanato para los hijos de los caídos, levantaría,
cómo relata en la fuente, un gran hospital para enfermos, incluyendo
salas de `internación'.

\textit{``
	En el sector que existe junto a la acrópolis, donde se abre
	el acceso al mar, había encontrado un templo de enorme tamaño
	bajo la advocación del gran apóstol Pablo, y construyo allí,
	dentro de la ciudad imperial, otra ciudad. [...] En su interior
	hay erigidas circularmente un conjunto abigarrado de viviendas,
	moradas para los más pobres y, lo que demuestra mayor caridad,
	hospicios para personas mutiladas.
	''}

\textit{``
	Este recinto circular es doble y gemelo. Los unos, hombre y
	mujeres mutilados, habitan en la parte superior. Otros se arratran
	en la planta baja.
	''}\footnote{\cite[p.~606]{alexiadaXV}}

La autora relata su experiencia viendo el lugar:

\textit{``
	Yo misma he llegado a ver a una mujer vieja asistida por una
	joven, a un hombre ciego guiado por manos de uno que sí ve,
	a personas sin pies que poseían pies, no lo suyos propios
	sino los de otros; a personas sin manos auxiliadas por las de
	otras personas, a criaturas recién nacidas amamantadas por otras
	madres, a paralíticos servidos por otros hombres robustos.
	Era doble la muchedumbre que recibía alimentos, pues unos se
	contaban entre los servidos y otros entre los servidores.
	''}\footnote{\cite[p.~607]{alexiadaXV}}

Finalizando con una descripción del lugar:

\textit{``
	Al templo del gran apóstol Pablo lo dotó con un importante y
	numeroso clero y con abundancia de lámparas. Si se visita este
	templo, se puede ver cómo cantan dos coros, uno a cada lado,
	alternativamente. Pues, como hizo Salomón, dispuso la existencia
	en el templo de los apóstoles de cantantes masculinos y femeninos.
	''}\footnote{\cite[p.~608]{alexiadaXV}}


































