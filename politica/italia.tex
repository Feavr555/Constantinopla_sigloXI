Para empezar a relatar sobre las relaciones políticas que tenía Bizancio,
me gustaría hacer un recuento de las fuentes que dispongo para analizar
su política y diplomacia.

%Segunda Revisión
En la fuente primaria “La Alexiada”, 
la fuente que inició este trabajo de investigación, 
dónde se relatan diversas campañas y alianzas 
que pueden darnos un gran panorama, 
%revisión 2
por otro lado el libro, un libro del historiador Roger Crowley, 
nos servirá para sustentar parte de lo que dice la fuente primaria. 
Pero, primordialmente, 
se va a emplear a “La Alexiada” como, prácticamente, 
única fuente del presente capítulo del trabajo de investigación.

%Revisión 2: pp.13
%Revisión 1: Eliminación de datos editoriales
Dentro de la fuente primaria,  Ana Comneno, “La Alexiada”,
\footnote{\cite[]{alexiadaVI}}
%en las páginas 249 a la 253,
se puede leer la alianza con Venecia (que más adelante provocaría 
la guerra entre Venecia y Génova que ahogarían a Bizancio), luego,
\footnote{\cite[pp.~419--460]{alexiadaX}}
%en las páginas 373 a la 291 y 419 a la 460,
se puede leer la 
primera cruzada, y por ende, la opinión de los romanos 
orientales de la misma. Cómo para resaltar,
\footnote{\cite[]{alexiadaV}}
%en las páginas 215 a la 218,
se hace referencia a la querella de las investiduras, 
pero la autora desconoce la causa real (por lo menos eso parece) 
y cree que la causa de que Alemania (Sacro Imperio)
%invada al papa es otra…
avance con su ejército sobre italia y los estados
pontificios es otra...
y las invasiones normandas, 
turcas y de las tribus del nómadas. 

Finalizando este pequeño resumen empecemos a 
relatar sobre cada uno de estos asuntos para conocer el 
panorama político y diplomático del Imperio Romano de Oriente.

Empecemos por la alianza Bizantino-veneciana, y la primera 
invasión normanda. Para entender el contexto, en los libros 
IV, V y VII de la Alexiada, se relata la primer invasión normanda 
(en el libro XII la segunda), el la cual Roberto Guiscardo, 
el duque de Apulia y futuro rey de Sicilia busca nuevas tierras de 
conquista en Grecia, dando inicio a la guerra. Aquí se puede ver el 
primer aspecto diplomático, hay una guerra entre el imperio bizantino 
y sicilia. En esta guerra, el emperador del momento, Alejo, 
como lo narra la fuente, llama a Venecia (que prácticamente 
era como una colonia de Bizancio al comienzo de su historia), 
la cual derrota en el mar a los normandos.

\textit{“Roberto se hizo cargo de toda la flota [normanda] y, 
[…] desde donde zarparía en dirección al Ilírico.[…]}

\textit{Roberto se percató de la ofensiva que pretendía la escuadra 
en contra 
de él y, anticipándose a la batalla de acuerdo con su carácter, soltó 
amarras y con toda su flota arribó al puerto Casope. Los venecianos, 
a su vez, llegaron al puerto de Pasaron y aguardaron allí un cierto 
tiempo. Cuando se enteraron de la llegada de Roberto, marcharon 
rápidamente también ellos al puerto de Casope. Tras un violento 
combate y un enfrentamiento al abordaje, Roberto fue derrotado. 
No por ello se rindió después de esta derrota, habida cuenta de su 
temperamento belicoso y dispuesto para el combate, sino que de nuevo 
se preparaba para luchar en otra batalla y enfrentarse en un 
combate más trascendente. Al conocer esto, los comandantes de ambas 
flotas, animados por la victoria, lo atacaron tres días después y 
lograron una brillante victoria sobre él. Luego, regresaron de 
nuevo al puerto de Pasaron.[…]
}
\textit{Transcurrido un tiempo, los venecianos aparejaron dromonos, 
tirremes y algunas otras naves pequeñas y veloces, y se 
encaminaron con mayores fuerzas contra Roberto.”
}\footnote{\cite[pp.~250--252]{alexiadaVI}}

\textit{“Él [el emperador de Bizancio] les correspondió con 
abundantes presentes y honores.
Honró al dux de Venecia con la dignidad de protosebasto junto con 
sus rentas. Honró también al patriarca con la dignidad de hipértimo en 
unión de sus correspondientes rentas. Igualmente, ordenó que anualmente 
fuera distribuida entre todas las iglesias de Venecia una importante 
cantidad de oro procedente del tesoro imperial. Hizo tributarios a todos los 
naturales de Melfi que poseyeran negocios en Constantinopla de la 
iglesia del apóstol evangelista San Marcos y cedió la explotación de 
los negocios que se extendían desde el antiguo muelle de los hebreos 
hasta el lugar llamado Bigla, incluidos los muelles existentes dentro 
de estos límites. Les regaló asimismo muchos inmuebles en la ciudad 
imperial, en Dirraquio y en donde se les antojase pedirlos. Y, lo que 
es más importante, les concedió la exención de aranceles en el comercio 
dentro de las fronteras del Imperio de los romanos, para que comerciasen 
libremente a voluntad, sin tener que aportar ni un óbolo en virtud de tasas 
comerciales o de cualquier clase de impuesto exigido para los fondos 
públicos, así como la dispensa de subordinarse a ninguna autoridad romana.”
}\footnote{\cite[p.~252]{alexiadaVI}}

El historiador Roger Crowley expone brevemente la relación 
que tenían estos dos:

\textit{“La relación entre Bizancio y Venecia fue intensa, compleja y longeva; 
también fue complicada, debido a sus visiones contradictorias del mundo, 
y estuvo sujeta a brutales cambios de humor. Sin embargo, Venecia siempre 
miró hacia Constantinopla. Era la gran ciudad del mundo, la puerta de 
entrada a oriente. A través de sus almacenes en el cuerno de oro fluía 
la riqueza del ancho mundo: pieles, cera, esclavos y caviar de Rusia; 
especias de la India y China, marfil, seda, piedras preciosas y oro. 
A partir de estos materiales los artesanos bizantinos creaban objetos 
extraordinarios, tanto sagrados como profanos -relicarios, mosaicos, 
cálices con esmeraldas incrustadas o vestidos de seda tornasolada- 
que conformaban el gusto veneciano. La asombrosa basílica de San Marcos, 
re consagrada en 1094, fue diseñada por arquitectos griegos siguiendo la 
pauta de la iglesia matriz de los santos apóstoles en Constantinopla; 
sus artesanos relataron la historia de San Marcos, piedra a piedra, 
imitando el estilo de los mosaicos de Santa Sofía; sus Orfebres y 
esmaltadores crearon la Pala D’Oro, el retablo dorado, una expresión 
milagrosa de la devoción y el arte bizantinos. 
El aroma de las especias de los muelles de Venecia había sido 
transportado mil millas, desde los almacenes del cuerno de oro. 
Constantinopla era el zoco de Venecia, donde sus mercaderes se reunían y
 ganaban (o perdían) fortunas. Cómo leales súbditos del emperador, 
 el derecho a comerciar en sus tierras fue siempre su posesión 
 más preciada. El imperio, a su vez, lo utilizó como elemento de 
 negociación para mantener a raya a sus arrogantes vasallos. 
 En 991, Orseolo consiguió valiosos derechos de comercio a cambio 
 de apoyo veneciano en el Adriático, veinticinco años después, 
 estos derechos le fueron retirados con enfado tras una disputa.”
 }\footnote{\cite[pp.~39--40]{venecia}}












