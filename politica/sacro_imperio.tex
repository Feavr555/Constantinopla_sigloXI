

\subsubsection{La Querella de las investiduras}

Durante este período del siglo XI, vivieron varios personajes
reconocidos de la Cristiandad, como el cid campeador, Guillermo
el conquistador de Inglaterra, Enrique IV de Alemania, además
de la primera cruzada, tema que trataré.

Cómo se pudo leer, está Enrique IV, nombrado en la Alexiada;
donde quisiera presentar la siguiente cita:

\textit{``[...] Había vuelto a enviar al rey de Alemania embajadores
que presidía el llamado Metimnes y por carta instaba con energía a no
retrasar más la ayuda y a invadir con gran rapidez Longibardía al frente
de sus tropas y contigentes extranjeros y poder expulsarlo del Ilírico.
A continuación le comunicó al rey de Alemania que, si obraba de este
modo, le quedaría muy reconocido y le aseguraba que la boda prometida
sería ultimada a través de los embajadores enviados por él.''}
\footnote{\cite[p.~215]{alexiadaV}}

En primer lugar, la autora hace referencia a Italia como ``Longibardía'',
en segundo lugar, la mención de Enrique IV como rey de Alemania, en
lugar de emperador. A la vez de referirse a si mismos cómo
``imperio de los romanos'' (en parte al ser de los romanos, y no
de Roma, ya que no la tenían...). Por lo cual, se puede ver
que se consideraban ellos como ``El imperio'' y no admitían que
Alemania lo fuese, ni siquiera como homólogo occidental.

Pues, en esa carta, el emperador Alejo I insta a Enrique IV a invadir
Italia, para distraer al rey de Sicilia, Roberto Guiscardo,
coincidiendo con la querella de las investiduras, donde Enrique IV
depone al Papa, cómo desarrollaré a continuación, siguiendo el libro de
``Ciudad cristiana'' de Rubén Calderon Bouchet, ya que a su vez
el Papa estaba aliado a Roberto Guiscardo, y fue él quien le coronó
rey de Sicilia.

\subsubsection{Desde otro punto de vista}

% Segunda revisión.

En la siguiente cita de Rubén Calderon Bouchet se desarrolló
desde el otro lado. El rey de Alemania, Enrique IV
se dirigió a Roma a deponer al Papa y poner su antipapa.
A su vez, el Papa Gregorio llamó a los normandos,
%que es Roberto Guiscardo, rey de Sicilia por el mismo papa Gregorio,
% -> cambio de renglón
o sea a Roberto Guiscardo quien habia sido coronado rey de Sicilia
por el mismo Papa Gregorio.
Guiscardo tuvo que dejar en manos de si hijo, Bohemundo de Tarento
(el mismo que sería príncipe de Antioquia en la primer cruzada)
en Grecia, mientras él respondía al llamado del Papa
contra Enrique IV.

Mientras que el emperador Alejo I del imperio Bizantino le había mandado
una carta, con una oferta de boda con una princesa bizantina,
si Enrique IV atacaba a los normandos.

¿Por qué atacó el emperador Enrique IV, por deponer al Papa Gregorio,
por la alianza con Alejo I, o ambos?

¡Una escena digna del príncipe de Maquiavelo!

\textit{
	``Enrique IV nuevamente dueño de la situación en Alemania por
	un oportuno triunfo militar convocó a un concilio a todos
	los obispos adictos y volvió a deponer al papa Gregorio
	y designó en su lugar a Wilberto de Ravena. Sin perder
	el tiempo marchó sobre Roma y colocó a su antipapa al frente
	de la cristiandad con el nombre de Clemente III.
	En la capital del cristianismo recibió de manos de Clemente
	la corona imperial.
}

\textit{
	``Gregorio no reconoció la autoridad de Enrique y luego
	de haber lanzado sobre él una nueva excomunión sin éxito
	se refugió en el castillo de Sant'Angelo y desde allí
	pidió ayuda militar a los normandos.
}

% -> ver en la fuente.
\textit{
	``Estos aventureros apenas desbastados de sus antecedentes
	bárbaros eran un arma de doble filo. Combatieron contra
	Enrique IV, tomaron la ciudad de Roma y, después de librar
	al Papa de su prisión en Sant'Angelo, sometieron la ciudad
	pontificia a un prolijo saqueo. Con este acto de bandolerismo
	se enajeraron la adhesión de los romanos que prefirieron
	en adelante la protección del excomulgado y su antipapa.
}

\textit{
	``Gregorio buscó refugio en la abadía de Montecassino y luego
	se dirigió a Salerno donde renovó la excomunión contra
	Enrique y Clemente y dirigió a todos los fieles una carta
	encíclica para exponerles la situación de la Iglesia.
}

\textit{
	``El asunto afectaba a toda la cristiandad en sus dos cabezas.
	En 1085 se reunióun concilio para examinar el problema
	de las excomuniones. Una y otra potestad tuvo en la
	asamblea sus defensores y sus impugnadores y la prolongación
	de la disputa reveló su completa esterilidad. El 25 de mayo de
	ese año murió Gregorio VII en Salerno.''
}\footnote{\cite[p.~871--872]{ciudad}}

\subsubsection{Relación entre ambos imperios}

Continuemos viendo la relación entre el sacro imperio y
el imperio bizantino. Ambos se consideraban imperio romano.
Los emperadores alemanes decían que no pueden llamarse romano
si no tienen Roma, pero Constantinopla se decía ``de los
romanos''.

\textit{``El hijo de Luis el Piadoso, también de nombre Luis,
	pero conocido como el Germánico, en una carta al
	emperador de Bizancio Basilio I, le explica el uso del
	título de Emperador de los Romanos que parece sorprender
	no poco al bizantino.}

\textit{``Conviene que sepas que si no fuéramos emperador de los
	romanos tampoco lo seríamos de los francos. Hemos recibido
	ese nombre y ese título de los mismos romanos, porque
	entre ellos brilló con viva claridad la cima de tan gran
	sublimidad y de tan prestigiosa apelación.
	La decisión de Dios nos hizo asumir el gobierno del pueblo
	y de la ciudad, así como la defensa y exaltación de la madre
	de todas las Iglesias, y es ella quien ha conferido a los
	primeros príncipes de nuestra dinastía primero la autoridad
	real y luego la imperial...
	Es por la unción que le confió el sumo pontífice que
	nuestro antepasado Carlos el Grande, primero de nuestra
	nación y de nuestra familia, ha sido declarado emperador
	y se ha convertido en el Cristo del Señor en razón
	de su piedad inmensa''.''}
\footnote{\cite[p.~846]{ciudad}}

Como cuenta y cita Bouchet, el conflicto entre ambos imperios
sobre referirse a si mismos como ``imperio de los romanos''.
A pesar de que en ese caso es entre Luis el gérmanico y
Basilio I, la princesa bizantina Ana Comnena, en su libro
la Alexiada, menciona al Sacro Imperio como reino de Alemania,
y al imperio de Constantinopla como Imperio de los Romanos.

\textit{
	````Así como en razón de nuestra fe en Cristo pertenecemos
	a la raza de Abraham... así hemos recibido el gobierno del
	Imperio Romano, en razón de nuestra buena ortodoxia.
	Los griegos, por el contrario, por sus opiniones heréticas,
	han dejado de ser emperadores de los romanos''.}

% -> Última de la segunda revisión
\textit{
	``Folz une los elementos esenciales de este documento
	y determina la idea carolingia del imperio
	en cinco puntos:\\
	a. Paridad del Imperio Romano y el Bizantino. Sin
	tendencia universalista.\\
	b. Se niega al emperador bizantino el título de emperador
	de los romanos porque no corresponde a una realidad política.\\
	c. Sublimidad del imperio, porque su dignidad reside cerca de
	Dios en el fuste de la piedad.\\
	d. La fuente de legitimación del imperio está en Roma.
	e. Los francos son portadores del título imperial por su
	ortodoxia y este imperio de substancia germánica es una a
	pesar de su división en reinos.''''}
\footnote{\cite[p.~846-847]{ciudad}}

Con esto se puede ver el conflicto de ambos imperios por
quien tenía que llamarse de los romanos. Siendo ambos los
que se lo adjudicaban.


















