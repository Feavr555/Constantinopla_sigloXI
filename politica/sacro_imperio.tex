

\subsubsection{La Querella de las investiduras}

Durante este período del siglo XI, vivieron varios personajes
reconocidos de la Cristiandad, como el cid campeador, Guillermo
el conquistador de Inglaterra, Enrique IV de Alemania, además
de la primer cruzada, tema que trataré.

Cómo se pudo leer, está Enrique IV, nombrado en la Alexiada;
donde quisieara presentar la siguiente cita:

\textit{``[...] Había vuelto a enviar al rey de Alemania embajadores
que presidía el llamado Metimnes y por carta instaba con energía a no
retrasar más la ayuda y a invadir con gran rápidez Longibardía al frente
de sus tropas y contigentes extranjeros y poder expulsarlo del Ilírico.
A continuación le comunicó al rey de Alemania que, si obraba de este
modo, le quedaría muy reconocido y le aseguraba que la boda prometida
sería ultimada a través de los embajadores enviados por él.''}
\footnote{\cite[p.~215]{alexiadaV}}

En primer lugar, la autora hace referencia a Italia como ``Longibardía'',
en segundo lugar, la mención de Enrique IV como rey de Alemania, en
lugar de emperador. A la vez de referirse a si mismos cómo
``imperio de los romanos'' (en parte al ser de los romanos, y no
de Roma, ya que no la tenían...). Por lo cual, se puede ver
que se consideraban ellos como ``El imperio'' y no admitían que
Alemania lo fuese, ni siquiera como homologo occidental.

Pues, en esa carta, el emperador Alejo I instá a Enrique IV a invadir
Italia, para distraer al rey de Sicilia, Roberto Guiscardo,
coincidiendo con la querella de las investiduras, donde Enrique IV
quita al papa, cómo desarrollaré a continuación del libro de
``Ciudad cristiana'' de Rubén Calderon Bouchet, ya que a su vez
el papa estaba aliado a Roberto Guiscardo, y fue él quien le corono
rey de Sicilia.

\subsubsection{Relación entre ambos imperios}

Continuemos viendo la relación entre el sacro imperio y
el imperio bizantino. Ambos se consideraban imperio romano.
Los emperadores alemanes decían que no pueden llamarse romano
si no tienen Roma, pero Constantinopla se decía ``de los
romanos''.

\textit{``El hijo de Luis el Piadoso, también de nombre Luis,
	pero conocido como el Germánico, en una carta al
	emperador de Bizancio Basilio I, le explica el uso del
	título de Emperador de los Romanos que parece sorprender
	no poco al bizantino.}

\textit{``Conviene que sepas que si no fuéramos emperador de los
	romanos tampoco lo seríamos de los francos. Hemos recibido
	ese nombre y ese título de los mismos romanos, porque
	entre ellos brilló con viva claridad la cima de tan gran
	sublimidad y de tan prestigiosa apelación.
	La decisión de Dios nos hizo asumir el gobierno del pueblo
	y de la ciudad, así como la defensa y exaltación de la madre
	de todas las Iglesias, y es ella quien ha conferido a los
	primeros príncipes de nuestra dinastía primero la autoridad
	real y luego la imperial...
	Es por la unción que le confió el sumo pontífice que
	nuestro antepasado Carlos el Grande, primero de nuestra
	nación y de nuestra familia, ha sido declarado emperador
	y se ha convertido en el Cristo del Señor en razón
	de su piedad inmensa''.''}
\footnote{\cite[p.~846]{ciudad}}

Como cuenta y cita Bouchet, el conflicto entre ambos imperios
sobre referirse a si mismos como ``imperio de los romanos''.
A pesar de que en ese caso es entre Luis el gérmanico y
Basilio I, la princesa bizantina Ana Comnena, en su libro
la Alexiada, menciona al Sacro Imperio como reino de Alemania,
y al imperio de Constantinopla como Imperio de los Romanos.





















