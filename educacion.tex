
	Para empezar con la educación de Constantinopla, se va a dividir en dos partes principales, el rector de la universidad de Constantinopla, y el emperador del imperio. De forma que se va a poder estudiar el programa de estudio y que obra hizo el “estado”.

	Voy a empezar a desarrollar por la parte pedagógica, con la obra e ideas de Miguel Psellos, continuando por el emperador Comneno, ya que gobernaron durante el siglo XI. En el primero, se va a dividir en una presentación del personaje junto con su visión educativa y programa de materias. En el segundo, se mencionará la preocupación por la educación del emperador, del colegio de huérfanos.

\subsection{Miguel Psellos}
	En la presente sección, desarrollaré cómo era la educación en la Constantinopla del siglo XI, permaneciendo fiel al objetivo del presente trabajo de averiguar y detallar cómo era la vida en este determinado espacio-tiempo.

\subsubsection{Presentación}
	Para empezar, relataré brevemente la historia de este personaje. Miguel Psellos vivió desde 1018 hasta el 1096 después de cristo, admirador de la cultura helénica y de Platón, aunque más viendo era neoplatónico, siendo “[...]el importador de este peculiar neoplatonismo en el pensamiento bizantino [...]”.\footnote{\cite[p.~160]{filosofia}}

\subsubsection{Visión Pedagógica}
	Primero veremos el método de enseñanza o pedagogía de Psellos; “[...]el método de la enseñanza que daba en la Universidad de Constantinopla; hallaremos así indicios característicos de la idea de Psellos tenía de la filosofía y que refleja cuando dice: “Antes de dedicarme al estudio de la filosofía, estuve enamorado de la retórica”. […] Con ciertas variantes expresa en otras ocasiones la misma idea, como por ejemplo:“Yo combino la filosofía con la retórica y procuro expresarme de acuerdo con ambas y con la ayuda de cada una de ellas.” En otros términos: que reclama armonía de formas y contenido. “El cuidado del estilo -insiste- no es, en modo alguno, obstáculo para el mérito.””\footnote{\cite[p.~160]{filosofia}}

	Para ir viendo el estilo de pedagogía que quería dar en la Universidad de Constantinopla, también: “[...]La retórica ejerce sobre Psellos una especie de fascinación, por lo que dice: “La sustancia del retórico no es más que el resplandor del lenguaje, se que hable, o que escriba, o que enseñe la retórica, o que se ocupe su pensamiento con la filosofía…; esto es lo que hace que el retórico no sea ni ateniense ni espartano, sino ciudadano del universo; es como el relámpago y el trueno, cuyo fulgor y sonido se imponen por doquier.””.\footnote{\cite[p.~161]{filosofia}}

	Entrando más en el tema pedagógico de Miguel Psellos, continua el autor:

	“A pesar de su fascinación a la retórico, Psellos no se decide exclusivamente por ella al tratar del problema pedagógico que separaba a los profesores de retórica (o sofistas) de los filósofos. “Es preciso -escribe a cierto padre- que pongas a tu hijo en medio de las fuentes, la filosofía y la retórica, y que le hagas beber de ambas, de cada una oportunamente; si no lo procuras así, le convertirás en una inteligencia desprovista de lenguaje, si es que sólo bebe de la fuente filosófica; o resultará mera lengua sin inteligencia en el caso contrario.”\footnote{\cite[p.~161]{filosofia}}

	Aunque, tampoco -dice- alcanzan para formar a un hombre, sino que necesita también de política “[...]si no, ese hombre, con toda su retórica sólo será un 
	\selectlanguage{greek}
	χύμβαλον άλαλάζον.
	\selectlanguage{spanish}
	 Destacamos así un rasgo característico de Psellos, ya que ni la filosofía ni la retórica son para él un fin en sí, sino meros instrumentos, de los que se sirve para educar a la juventud bizantina, a la que ha sabido inculcar su amor a Grecia.”\footnote{\cite[p.~161]{filosofia}}

	Por lo cual, puede verse el interés de él por ser un buen orador para enseñar.

	“Según otro punto de vista suyo, el valor de la filosofía y de la retórica es muy relativo, si no nulo, ya que comparados con la Sagrada Escritura, los retóricos y los filósofos, los caldeos y los egipcios, todos son como el bronce comparado con el oro”.\footnote{\cite[p.~161]{filosofia}}

\subsubsection{Programa}
	En cuanto lo visto anteriormente, Psellos desarrollo su programa para la Universidad de Constantinopla; donde se vería retórica, dialéctica, el trivium, aritmética, geometría, música, astronomía y el quadrivium.

	“Fiel a estas ideas, Psellos estableció en la Universidad de Constantinopla el programa siguiente: Primero se instruía a los principiantes en retórica y en dialéctica; terminado el trivium, los estudiantes pasaban al curso superior, el de Psellos, en el que estudiaban aritmética, geometría, música, astronomía, las ciencias enumeradas en el libro IV de la República de Platón y el quadrivium. Pero aquí no pasaban a la dialéctica (como quería Platón, ni en el sentido que él preconizaba), sino que seguían con los estudios de filosofía, por considerar a ésta como el complemento de todas las ciencias. Desde luego, el estudio de la filosofía (que se comenzaba por la lógica de Aristóteles) no constituía un nuevo grado, sino que era más bien la continuación y culminación del quadrivium.”\footnote{\cite[pp.~161--162]{filosofia}}

	Dentro de la universidad, la cátedra de filología la tenía un tal Nicetas, el cual usaba un tal método de ciencias para la gramática proveniente de Alejandría:

	“Conviene señalar que el curso de gramática comprendía la filología entra; el estudio de los diversos dialectos griegos hacía resaltar más las leyes principales de esta lengua; la ley de analogía, que se aplicaba sobre todo para explicar la declinación, ayudaba a comprender la lengua en su regularidad racional. Las teorías rítmicas y musicales venían a completar el estudio del lenguaje. En resumen: que Nicetas usaba ampliamente la ciencia de los gramáticos de Alejandría:”\footnote{\cite[p.~162]{filosofia}}

	Miguel Psellos, era un personaje religioso, que el autor lo marca lejos de tener una ciencia positiva, al mirar las ciencias y mitología griega desde las Sagradas Escrituras.


\subsubsection{Retórica}
	Ya que hemos empezado a adentrándonos en la filología, gramática y retórica anteriormente, con los métodos de enseñanza de Psellos y Nicetas, continuemos con las tres principales materias del estudio, la retórica/oratoria, la filosofía y la metafísica.

	A continuación, continuaré con oratoria:

	“No solo Nicetas de ocupaba de la filología, pues también Psellos enseñaba gramática, si bien sus preferencias tendían al examen de las leyes que rigen el mundo de la oratoria, procurando que sus discípulos adquirieran cualidades estilísticas. Su examen teórico iban seguidos de ejercicios prácticos, más lecturas y explicaciones de páginas de oradores y de poetas, elegidas unas y otras por la perfección de su forma y el interés de su contenido. Se cuenta que, para tal examen, Psellos seguía a Hermógenes de Tarso; pero aunque no hay duda de que conocía los escritos de éste y se servía de ellos, su objetivo era (según declara formalmente) enseñar la retórica antigua, “la que el propio Platón prefería”, una retórica que no gustaba a sus discípulos (quienes preferían la de Hermógenes) y para la cual se guiaba siempre por Platón y Aristóteles.”\footnote{\cite[p.~163]{filosofia}}

\subsubsection{Quadrivium}
	En cuanto a este curso, el autor relata los cursos que se enseñaban en el quadrivium.

	“Para enseñar las ciencias del quadrivium, Psellos utilizó los manuales de Nicómaco de Gerasa, de Euclides, de Diofante y de Teón de Esmirna para las matemáticas, los de Tolomeo y Proclo para la astronomía y los de Aristógenes para la música. Al tratar de las ciencias físicas y naturales, insiste particularmente en las teorías de lo frío y de lo cálido, lo seco y lo húmedo, la explicación de los fenómenos meteorológicos y de los terremotos, así como la descripción de la Tierra. También se incluye en sus enseñanzas la adivinación, la astrología y las narraciones maravillosas, las que pretende explicar aclarando que los hechos ocurridos en esos dominios no son extraordinarios sino en apariencia, ya que en su fondo están regidos por las leyes científicas."\footnote{\cite[p.~163]{filosofia}}

\subsubsection{Filosofía}
	Este era un curso, dividido en lógica y metafísica.

	“El objetivo del curso de filosofía era dar, mediante la lógica y la metafísica aristotélicas, los fundamentos del pensamiento filosófico, a la vez que servía de punto de partida para el estudio de los problemas especulativos. Consideraba a la filosofía como una etapa preparatoria de la metafísica, cuyo material era tomado casi íntegramente de Plotino, Proclo y Platón, y no de Aristóteles. Las doctrinas filosóficas de la metafísica tenían a la teología, filosofía primera, como referencia máxima, y a la luz de ella debían los estudiantes interpretar los textos teológicos. Tras esto se puede comprender en todo su valor el sentido de la definición de Psellos daba a sus lecciones, de las que: “La lección es la perfección del alma, su redacción y su ascensión, o su vuelta al bien supremo."\footnote{\cite[pp.~164--165]{filosofia}}

\subsubsection{Metafísica}
	En cuanto a la última materia que desarrollá Tatakis, es la Metafísica, en la cual pone;

	“Además de la filosofía, constituían asignaturas de la etapa preparatoria de la metafísica la historia de la filosofía, la exégesis de las narraciones legendarias y de las tradiciones y sentencias populares… siempre que no se utilizan como argumentos para probar la verdad de los resultados adquiridos. La historia de la filosofía no se limitaba  a relatar el pensamiento griego, sino que abarcaba también las doctrinas provenientes del Oriente (de Caldea, Egipto y Palestina) e inclusive el pensamiento hermético. Por otra parte, el análisis de las tradiciones populares y de las narraciones legendarias de la Antigüedad conseguía extraer de ellas ideas sublimen que Psellos trasladaba al cristianismo. Este programa de estudios muestra el papel importantísimo, central, que se le había reservado a la tradición helénica; de ella se esperaba no sólo que modelara el justo y el juicio de la juventud, sino que adornará además su alma con valores morales; y notemos bien que Psellos procura mostrar esta tradición helenística tanto en sus aciertos como en sus imperfecciones, considerándola siempre como precursora del cristianismo.”\footnote{\cite[p.~165]{filosofia}}

\subsection{Ana y Alejo Comneno}
