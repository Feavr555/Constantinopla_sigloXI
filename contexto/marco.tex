
%El contexto científico en el que se va a encontrar el trabajo, 
%se trata de recopilar información de los diversos estudios que 
%hablen del tema elegido. Ya que el tema elegido (cómo se vivía en la 
%Constantinopla del siglo XI) no encontre nadie que trate de 
%forma directa el ``cómo se vivía'' como título o 
%interrogante principal, los antecedentes del estudio son los que 
%investigaron brindando la información necesaria para la reconstrucción 
%de cómo pudo ser la vida en la Constantinopla del siglo XI, y los 
%novelistas, que estudiando y leyendo para la construcción histórica 
%del escenario de sus novelas, estudiaron el tema de forma directa, 
%como es el objetivo del trabajo.


%Primer revisión, cambio de parrafo
Me propongo analizar la vista cultural, social, militar y educativa de
Constantinopla durante el siglo XI. Elegí ese siglo por
la lectura del libro de la Alexiada en 2019, citada en el
presente trabajo.
En este trabajo me propongo recopilar información de diversos
estudios en los que se aborda el tema elegido.
Esas investigaciones me brindan la información necesaria para la
reconstrucción de cómo pudo ser la vida en la Constantinopla
del siglo XI. A esto se agrega, un trabajo que pueda servirme
y pueda servirle a novelistas para la construcción del escenario
de sus novelas.

%Nuevo parrafo, tras revisión 1
El tema central del trabajo es la reconstrucción de la vida
en dicha ciudad y tiempo, tal como se enuncia en la primer
interrogante.

Interrogantes e hipótesis del trabajo:

¿Cómo se vivía en la Constantinopla del siglo XI?

%Parentesis a la interrogante
¿Cuál era el plan de estudios?
(De la Universidad de Constantinopla, y la preocupación
por la educación del emperador Alejo (último libro))

¿Cómo era la salud pública?

¿Cómo se organizaba el ejército?

¿Quiénes eran los personajes político-militares-culturales?

A pesar de que varias de las anteriores interrogantes ya las habrán 
tratado de forma directa, en este trabajo se recopilaría dicha 
información con el fin de reconstruir la vida de Constantinopla del 
siglo XI, como figura anteriormente.

%Agregado autor
Por ejemplo, en el libro XV de “La Alexiada”, 
de Ana Comneno traducido por Emilio Díaz
Ronaldo, se relata un gran hospital 
público, y un colegio, publico, para huérfanos, junto con una nueva 
didáctica de dicho colegio. Además, se relata una nueva formación militar.

%Agregado autor
Y, en el capítulo IV de “Filosofía Bizantina”, 
por Basilio Tatakis, se describe el plan de 
estudios que había diseñado el catedrático de la universidad de 
Constantinopla, Miguel Psellos.

El método que se va a usar en el trabajo, es el método bibliográfico. 
Ya que se va a confiar en los testimonios de los que estudiaron el 
tema, y en la fuente primaria de “La Alexiada”.


