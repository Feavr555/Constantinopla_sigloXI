\documentclass[a4paper,13pt,titlepage,oneside]{article}
\usepackage[greek,spanish]{babel}
\usepackage{setspace}
\usepackage[utf8]{inputenc}
\usepackage{times}
\usepackage{graphicx}
\usepackage{apacite}
\usepackage{csquotes}
\usepackage[T1]{fontenc}
\usepackage{anyfontsize}
\usepackage{fancyhdr}
\pagestyle{fancy}


%\textbf{...}		-> negrita
%\textit{...}		-> cursiva / italica
%\underline{...}	-> subrayado


%\fancyhf{}
%\lhead[\leftmark]{Marco Debiasi}
%\rhead[Constantinopla en el siglo XI]{\rightmark}
%\lfoot[\thepage]{}
%\rfoot[]{\thepage} 

\bibliographystyle{apacite}

\setcounter{secnumdepth}{0}
%\setcounter{tocdepth}{1}

\newcounter{ns}
\addtocounter{ns}{1}

%\title{Constantinopla en el siglo XI}
%\author{Marco Patricio Debiasi}
%\date{\today}
%\pagestyle{myheadings}
%\markright{Marco P. Debiasi}

\begin{document}
	\begin{titlepage}
	\centering{\fontsize{40pt}{48pt}
	\selectfont Constantinopla en el siglo XI \par}
	\vspace{1.5cm}

	\centering{\fontsize{22pt}{20pt}\selectfont\includegraphics[width=0.7\linewidth]{ico.png}}

	\fontsize{22pt}{20pt}\selectfont Marco Patricio Debiasi\par
	\vspace{1cm}
	\fontsize{22pt}{10pt}\selectfont Profesorado de Historia\par
	\end{titlepage}
\spacing{1.5}


\newpage

\section{Marco de Referencia}

%El contexto científico en el que se va a encontrar el trabajo, 
%se trata de recopilar información de los diversos estudios que 
%hablen del tema elegido. Ya que el tema elegido (cómo se vivía en la 
%Constantinopla del siglo XI) no encontre nadie que trate de 
%forma directa el ``cómo se vivía'' como título o 
%interrogante principal, los antecedentes del estudio son los que 
%investigaron brindando la información necesaria para la reconstrucción 
%de cómo pudo ser la vida en la Constantinopla del siglo XI, y los 
%novelistas, que estudiando y leyendo para la construcción histórica 
%del escenario de sus novelas, estudiaron el tema de forma directa, 
%como es el objetivo del trabajo.


%Primer revisión, cambio de parrafo
Me propongo analizar la vista cultural, social, militar y educativa de
Constantinopla durante el siglo XI. Elegí ese siglo por
la lectura del libro de la Alexiada en 2019, citado en el
presente trabajo.
En este trabajo me propongo recopilar información de diversos
estudios en los que se aborda el tema elegido.
Esas investigaciones me brindan la información necesaria para la
reconstrucción de cómo pudo ser la vida en la Constantinopla
del siglo XI. 
%A esto se agrega, un trabajo que pueda servirme
%y pueda servirle a novelistas para la construcción del escenario
%de sus novelas. 
%Segunda revision
A esto de agrega, la idea que de este trabajo pueda servirme
y que pueda servirle a novelistas para la construcción del escenario
de sus novelas.

%Nuevo parrafo, tras revisión 1
El tema central del trabajo es la reconstrucción de la vida
en dicha ciudad y tiempo, tal como se enuncia en la primer
interrogante.

Interrogantes e hipótesis del trabajo:

¿Cómo se vivía en la Constantinopla del siglo XI?

%Parentesis a la interrogante
¿Cuál era el plan de estudios?
(De la Universidad de Constantinopla, y la preocupación
por la educación del emperador Alejo (último libro))

¿Cómo era la salud pública?

¿Cómo se organizaba el ejército?

¿Quiénes eran los personajes político-militares-culturales?

%Segunda revisión
A pesar de que varias de los anteriores interrogantes ya los habrán 
tratado de forma directa, en este trabajo se recopilaría dicha 
información con el fin de reconstruir la vida de Constantinopla del 
siglo XI, como figura anteriormente.

%Agregado autor
Por ejemplo, en el libro XV de “La Alexiada”, 
de Ana Comneno traducido por Emilio Díaz
Ronaldo, se describe un gran hospital 
público, y un colegio, publico, para huérfanos, junto con una nueva 
didáctica de dicho colegio. Además, se caracteriza una nueva formación militar.

%Agregado autor
Y, en el capítulo IV de “Filosofía Bizantina”, 
por Basilio Tatakis, se describe el plan de 
estudios que había diseñado el catedrático de la universidad de 
Constantinopla, Miguel Psellos.

El método que se va a usar en el trabajo, es el método bibliográfico. 
Ya que se va a confiar en los testimonios de los que estudiaron el 
tema, y en la fuente primaria de “La Alexiada”.



\newpage

\section{Introducción}
En el presente trabajo se desarrollará varios aspectos de cómo 
pudo haber sido la vida en la Constantinopla del siglo XI; viendo 
su filosofía, cultura, educación, salud y política. 
%Cuya organización está hecha para que cada capítulo 
%sea independiente 
%de los demás, para que se pueda leer como si fueran un 
%conjunto de artículos.

%Modificación de la primer modificación
La organización de este trabajo responde a los diferentes
subtemas que componen la reconstrucción de la vida social y
cultural de Constantinopla en el siglo elegido.



\newpage

\section{Capítulo 1: Filosofía Bizantina}
\subsection{La Mística}
	\subsubsection{San Simeón, el nuevo teologo}
	\input{filosofia/san_simeon}
	\subsubsection{Calixto Kataphigiotis}
	\input{filosofia/calixto}
\subsection{La Moral}
	\input{filosofia/moral}
	\subsubsection{Cecaumeno}
	\subsubsection{Nikoulitzas}
	\subsubsection{Teofilacto de Bulgaria}
	\subsubsection{Neoplatonismo}
	\input{filosofia/neoplatonismo}


\newpage

\section{Capítulo 2: Cultura}
\input{cultura/cultura}
\newpage

\section{Capítulo 3: Educación}

	Para empezar con la educación de Constantinopla, se va a dividir en dos partes principales, el rector de la universidad de Constantinopla, y el emperador del imperio. De forma que se va a poder estudiar el programa de estudio y que obra hizo el “estado”.

	Voy a empezar a desarrollar por la parte pedagógica, con la obra e ideas de Miguel Psellos, continuando por el emperador Comneno, ya que gobernaron durante el siglo XI. En el primero, se va a dividir en una presentación del personaje junto con su visión educativa y programa de materias. En el segundo, se mencionará la preocupación por la educación del emperador, del colegio de huérfanos.

\subsection{Miguel Psellos}
	En la presente sección, desarrollaré cómo era la educación en la Constantinopla del siglo XI, permaneciendo fiel al objetivo del presente trabajo de averiguar y detallar cómo era la vida en este determinado espacio-tiempo.

\subsubsection{Presentación}
	Para empezar, relataré brevemente la historia de este personaje. Miguel Psellos vivió desde 1018 hasta el 1096 después de cristo, admirador de la cultura helénica y de Platón, aunque más viendo era neoplatónico, siendo “[...]el importador de este peculiar neoplatonismo en el pensamiento bizantino [...]”.\footnote{\cite[p.~160]{filosofia}}

\subsubsection{Visión Pedagógica}
	Primero veremos el método de enseñanza o pedagogía de Psellos; “[...]el método de la enseñanza que daba en la Universidad de Constantinopla; hallaremos así indicios característicos de la idea de Psellos tenía de la filosofía y que refleja cuando dice: “Antes de dedicarme al estudio de la filosofía, estuve enamorado de la retórica”. […] Con ciertas variantes expresa en otras ocasiones la misma idea, como por ejemplo:“Yo combino la filosofía con la retórica y procuro expresarme de acuerdo con ambas y con la ayuda de cada una de ellas.” En otros términos: que reclama armonía de formas y contenido. “El cuidado del estilo -insiste- no es, en modo alguno, obstáculo para el mérito.””\footnote{\cite[p.~160]{filosofia}}

	Para ir viendo el estilo de pedagogía que quería dar en la Universidad de Constantinopla, también: “[...]La retórica ejerce sobre Psellos una especie de fascinación, por lo que dice: “La sustancia del retórico no es más que el resplandor del lenguaje, se que hable, o que escriba, o que enseñe la retórica, o que se ocupe su pensamiento con la filosofía…; esto es lo que hace que el retórico no sea ni ateniense ni espartano, sino ciudadano del universo; es como el relámpago y el trueno, cuyo fulgor y sonido se imponen por doquier.””.\footnote{\cite[p.~161]{filosofia}}

	Entrando más en el tema pedagógico de Miguel Psellos, continua el autor:

	“A pesar de su fascinación a la retórico, Psellos no se decide exclusivamente por ella al tratar del problema pedagógico que separaba a los profesores de retórica (o sofistas) de los filósofos. “Es preciso -escribe a cierto padre- que pongas a tu hijo en medio de las fuentes, la filosofía y la retórica, y que le hagas beber de ambas, de cada una oportunamente; si no lo procuras así, le convertirás en una inteligencia desprovista de lenguaje, si es que sólo bebe de la fuente filosófica; o resultará mera lengua sin inteligencia en el caso contrario.”\footnote{\cite[p.~161]{filosofia}}

	Aunque, tampoco -dice- alcanzan para formar a un hombre, sino que necesita también de política “[...]si no, ese hombre, con toda su retórica sólo será un 
	\selectlanguage{greek}
	χύμβαλον άλαλάζον.
	\selectlanguage{spanish}
	 Destacamos así un rasgo característico de Psellos, ya que ni la filosofía ni la retórica son para él un fin en sí, sino meros instrumentos, de los que se sirve para educar a la juventud bizantina, a la que ha sabido inculcar su amor a Grecia.”\footnote{\cite[p.~161]{filosofia}}

	Por lo cual, puede verse el interés de él por ser un buen orador para enseñar.

	“Según otro punto de vista suyo, el valor de la filosofía y de la retórica es muy relativo, si no nulo, ya que comparados con la Sagrada Escritura, los retóricos y los filósofos, los caldeos y los egipcios, todos son como el bronce comparado con el oro”.\footnote{\cite[p.~161]{filosofia}}

\subsubsection{Programa}
	En cuanto lo visto anteriormente, Psellos desarrollo su programa para la Universidad de Constantinopla; donde se vería retórica, dialéctica, el trivium, aritmética, geometría, música, astronomía y el quadrivium.

	“Fiel a estas ideas, Psellos estableció en la Universidad de Constantinopla el programa siguiente: Primero se instruía a los principiantes en retórica y en dialéctica; terminado el trivium, los estudiantes pasaban al curso superior, el de Psellos, en el que estudiaban aritmética, geometría, música, astronomía, las ciencias enumeradas en el libro IV de la República de Platón y el quadrivium. Pero aquí no pasaban a la dialéctica (como quería Platón, ni en el sentido que él preconizaba), sino que seguían con los estudios de filosofía, por considerar a ésta como el complemento de todas las ciencias. Desde luego, el estudio de la filosofía (que se comenzaba por la lógica de Aristóteles) no constituía un nuevo grado, sino que era más bien la continuación y culminación del quadrivium.”\footnote{\cite[pp.~161--162]{filosofia}}

	Dentro de la universidad, la cátedra de filología la tenía un tal Nicetas, el cual usaba un tal método de ciencias para la gramática proveniente de Alejandría:

	“Conviene señalar que el curso de gramática comprendía la filología entra; el estudio de los diversos dialectos griegos hacía resaltar más las leyes principales de esta lengua; la ley de analogía, que se aplicaba sobre todo para explicar la declinación, ayudaba a comprender la lengua en su regularidad racional. Las teorías rítmicas y musicales venían a completar el estudio del lenguaje. En resumen: que Nicetas usaba ampliamente la ciencia de los gramáticos de Alejandría:”\footnote{\cite[p.~162]{filosofia}}

	Miguel Psellos, era un personaje religioso, que el autor lo marca lejos de tener una ciencia positiva, al mirar las ciencias y mitología griega desde las Sagradas Escrituras.


\subsubsection{Retórica}
	Ya que hemos empezado a adentrándonos en la filología, gramática y retórica anteriormente, con los métodos de enseñanza de Psellos y Nicetas, continuemos con las tres principales materias del estudio, la retórica/oratoria, la filosofía y la metafísica.

	A continuación, continuaré con oratoria:

	“No solo Nicetas de ocupaba de la filología, pues también Psellos enseñaba gramática, si bien sus preferencias tendían al examen de las leyes que rigen el mundo de la oratoria, procurando que sus discípulos adquirieran cualidades estilísticas. Su examen teórico iban seguidos de ejercicios prácticos, más lecturas y explicaciones de páginas de oradores y de poetas, elegidas unas y otras por la perfección de su forma y el interés de su contenido. Se cuenta que, para tal examen, Psellos seguía a Hermógenes de Tarso; pero aunque no hay duda de que conocía los escritos de éste y se servía de ellos, su objetivo era (según declara formalmente) enseñar la retórica antigua, “la que el propio Platón prefería”, una retórica que no gustaba a sus discípulos (quienes preferían la de Hermógenes) y para la cual se guiaba siempre por Platón y Aristóteles.”\footnote{\cite[p.~163]{filosofia}}

\subsubsection{Quadrivium}
	En cuanto a este curso, el autor relata los cursos que se enseñaban en el quadrivium.

	“Para enseñar las ciencias del quadrivium, Psellos utilizó los manuales de Nicómaco de Gerasa, de Euclides, de Diofante y de Teón de Esmirna para las matemáticas, los de Tolomeo y Proclo para la astronomía y los de Aristógenes para la música. Al tratar de las ciencias físicas y naturales, insiste particularmente en las teorías de lo frío y de lo cálido, lo seco y lo húmedo, la explicación de los fenómenos meteorológicos y de los terremotos, así como la descripción de la Tierra. También se incluye en sus enseñanzas la adivinación, la astrología y las narraciones maravillosas, las que pretende explicar aclarando que los hechos ocurridos en esos dominios no son extraordinarios sino en apariencia, ya que en su fondo están regidos por las leyes científicas."\footnote{\cite[p.~163]{filosofia}}

\subsubsection{Filosofía}
	Este era un curso, dividido en lógica y metafísica.

	“El objetivo del curso de filosofía era dar, mediante la lógica y la metafísica aristotélicas, los fundamentos del pensamiento filosófico, a la vez que servía de punto de partida para el estudio de los problemas especulativos. Consideraba a la filosofía como una etapa preparatoria de la metafísica, cuyo material era tomado casi íntegramente de Plotino, Proclo y Platón, y no de Aristóteles. Las doctrinas filosóficas de la metafísica tenían a la teología, filosofía primera, como referencia máxima, y a la luz de ella debían los estudiantes interpretar los textos teológicos. Tras esto se puede comprender en todo su valor el sentido de la definición de Psellos daba a sus lecciones, de las que: “La lección es la perfección del alma, su redacción y su ascensión, o su vuelta al bien supremo."\footnote{\cite[pp.~164--165]{filosofia}}

\subsubsection{Metafísica}
	En cuanto a la última materia que desarrollá Tatakis, es la Metafísica, en la cual pone;

	“Además de la filosofía, constituían asignaturas de la etapa preparatoria de la metafísica la historia de la filosofía, la exégesis de las narraciones legendarias y de las tradiciones y sentencias populares… siempre que no se utilizan como argumentos para probar la verdad de los resultados adquiridos. La historia de la filosofía no se limitaba  a relatar el pensamiento griego, sino que abarcaba también las doctrinas provenientes del Oriente (de Caldea, Egipto y Palestina) e inclusive el pensamiento hermético. Por otra parte, el análisis de las tradiciones populares y de las narraciones legendarias de la Antigüedad conseguía extraer de ellas ideas sublimen que Psellos trasladaba al cristianismo. Este programa de estudios muestra el papel importantísimo, central, que se le había reservado a la tradición helénica; de ella se esperaba no sólo que modelara el justo y el juicio de la juventud, sino que adornará además su alma con valores morales; y notemos bien que Psellos procura mostrar esta tradición helenística tanto en sus aciertos como en sus imperfecciones, considerándola siempre como precursora del cristianismo.”\footnote{\cite[p.~165]{filosofia}}

\subsection{Ana y Alejo Comneno}

\newpage

\section{Capítulo 4: Salud}


Antes de empezar con el tema de salud, me gustaría aclarar 
que lo siguiente es sacado de la fuente primaria del 
presente trabajo, “La Alexiada”, del libro XV, el último de la obra. 
En el que se narra las obras que había mandado hacer el emperador, 
según lo narra su hija, a los huérfanos y demás gente afectada 
por las continuas guerras que estaba teniendo el imperio.

En el libro XV, los párrafos del 4 al 8 están dedicados a 
describir un hospital que el emperador había mandado a construir, 
siendo el único fragmento destinado a la salud dentro del libro 
(sin contar a la sanidad de campaña (militar) que se describe 
en el mismo libro XV, sección VII, párrafos 1 y 2).

\subsection{Sistema}

\subsection{Edificios}



\newpage

\section{Capítulo 5: Política}
\subsection{Relaciones con Italia}
Para empezar a relatar sobre las relaciones políticas que tenía Bizancio,
me gustaría hacer un recuento de las fuentes que dispongo para analizar
su política y diplomacia.

Empiezo por la fuente primaria “La Alexiada”, 
la fuente que inició este trabajo de investigación, 
dónde se relatan diversas campañas y alianzas 
que pueden darnos un gran panorama, 
por último, un libro del historiador Roger Crowley, 
para sustentar parte de lo que dice la fuente primaria. 
Pero, primordialmente, 
se va a poner a “La Alexiada” como, prácticamente, 
única fuente del presente capítulo del trabajo de investigación.

%Revisión 1
Dentro de la fuente primaria,  Ana Comneno, “La Alexiada”, 
Ático de los libros, Barcelona, 2016, en las páginas 249 a la 253, 
se puede leer la alianza con Venecia (que más adelante provocaría 
la guerra entre Venecia y Génova que ahogarían a Bizancio), luego, 
en las páginas 373 a la 291 y 419 a la 460, se puede leer la 
primera cruzada, y por ende, la opinión de los romanos 
orientales de la misma. Cómo para resaltar, en las páginas 
215 a la 218, se hace referencia a la querella de las investiduras, 
pero la autora desconoce la causa real (por lo menos eso parece) 
y cree que la causa de que Alemania (Sacro Imperio) invada al 
papa es otra… y las invasiones normandas, 
turcas y de las tribus del nómadas. 

Finalizando este pequeño resumen, pantallazo, empecemos a 
relatar sobre cada uno de estos asuntos para conocer el 
panorama político y diplomático del Imperio Romano de Oriente.

Empecemos por la alianza Bizantino-veneciana, y la primer 
invasión normanda. Para entender el contexto, en los libros 
IV, V y VII de la Alexiada, se relata la primer invasión normanda 
(en el libro XII la segunda), el la cual Roberto Guiscardo, 
el duque de Apulia y futuro rey de Sicilia busca nuevas tierras de 
conquista en Grecia, dando inició a la guerra. Aquí se puede ver el 
primer aspecto diplomático, hay una guerra entre el imperio bizantino 
y sicilia. En esta guerra, el emperador del momento, Alejo, 
como lo narra la fuente, llama a Venecia (que prácticamente 
era como una colonia de Bizancio al comienzo de su historia), 
la cual derrota en el mar a los normandos.

“Roberto se hizo cargo de toda la flota [normanda] y, 
[…] desde donde zarparía en dirección al Ilírico.[…]

Roberto se percató de la ofensiva que pretendía la escuadra en contra 
de él y, anticipándose a la batalla de acuerdo con su carácter, soltó 
amarras y con toda su flota arribó al puerto Casope. Los venecianos, 
a su vez, llegaron al puerto de Pasaron y aguardaron allí un cierto 
tiempo. Cuando se enteraron de la llegada de Roberto, marcharon 
rápidamente también ellos al puerto de Casope. Tras un violento 
combate y un enfrentamiento al abordaje, Roberto fue derrotado. 
No por ello se rindió después de esta derrota, habida cuenta de su 
temperamento belicoso y dispuesto para el combate, sino que de nuevo 
se preparaba para luchar en otra batalla y enfrentarse en un 
combate más trascendente. Al conocer esto, los comandantes de ambas 
flotas, animados por la victoria, lo atacaron tres días después y 
lograron una brillante victoria sobre él. Luego, regresaron de 
nuevo al puerto de Pasaron.[…]

Transcurrido un tiempo, los venecianos aparejaron dromonos, 
tirremes y algunas otras naves pequeñas y veloces, y se 
encaminaron con mayores fuerzas contra Roberto.”
\footnote{\cite[pp.~250--252]{alexiadaVI}}

“Él [el emperador de Bizancio] les correspondió con abundantes presentes y
honores. Honró al dux de Venecia con la dignidad de protosebasto junto con 
sus rentas. Honró también al patriarca con la dignidad de hipértimo en 
unión de sus correspondientes rentas. Igualmente, ordenó que anualmente 
fuera distribuida entre todas las iglesias de Venecia una importante 
cantidad de oro procedente del tesoro imperial. Hizo tributarios a todos los 
naturales de Melfi que poseyeran negocios en Constantinopla de la 
iglesia del apóstol evangelista San Marcos y cedió la explotación de 
los negocios que se extendían desde el antiguo muelle de los hebreos 
hasta el lugar llamado Bigla, incluidos los muelles existentes dentro 
de estos límites. Les regaló asimismo muchos inmuebles en la ciudad 
imperial, en Dirraquio y en donde se les antojase pedirlos. Y, lo que 
es más importante, les concedió la exención de aranceles en el comercio 
dentro de las fronteras del Imperio de los romanos, para que comerciasen 
libremente a voluntad, sin tener que aportar ni un óbolo en virtud de tasas 
comerciales o de cualquier clase de impuesto exigido para los fondos 
públicos, así como la dispensa de subordinarse a ninguna autoridad romana.”
\footnote{\cite[p.~252]{alexiadaVI}}

El historiador Roger Crowley expone brevemente la relación 
que tenían estos dos:

“La relación entre Bizancio y Venecia fue intensa, compleja y longeva; 
también fue complicada, debido a sus visiones contradictorias del mundo, 
y estuvo sujeta a brutales cambios de humor. Sin embargo, Venecia siempre 
miró hacia Constantinopla. Era la gran ciudad del mundo, la puerta de 
entrada a oriente. A través de sus almacenes en el cuerno de oro fluía 
la riqueza del ancho mundo: pieles, cera, esclavos y caviar de rusia; 
especias de la India y China, marfil, seda, piedras preciosas y oro. 
A partir de estos materiales los artesanos bizantinos creaban objetos 
extraordinarios, tanto sagrados como profanos -relicarios, mosaicos, 
cálices con esmeraldas incrustadas o vestidos de seda tornasolada- 
que conformaban el gusto veneciano. La asombrosa basílica de San Marcos, 
re consagrada en 1094, fue diseñada por arquitectos griegos siguiendo la 
pauta de la iglesia matriz de los santos apóstoles en Constantinopla; 
sus artesanos relataron la historia de San Marcos, piedra a piedra, 
imitando el estilo de los mosaicos de Santa Sofía; sus Orfebres y 
esmaltadores crearon la Pala D’Oro, el retablo dorado, una expresión 
milagrosa de la devoción y el arte bizantinos. 
El aroma de las especias de los muelles de Venecia había sido 
transportado mil millas, desde los almacenes del cuerno de oro. 
Constantinopla era el zoco de Venecia, donde sus mercaderes se reunían y
 ganaban (o perdían) fortunas. Cómo leales súbditos del emperador, 
 el derecho a comerciar en sus tierras fue siempre su posesión 
 más preciada. El imperio, a su vez, lo utilizó como elemento de 
 negociación para mantener a raya a sus arrogantes vasallos. 
 En 991, Orseolo consiguió valiosos derechos de comercio a cambio 
 de apoyo veneciano en el Adriático, veinticinco años después, 
 estos derechos le fueron retirados con enfado tras una disputa.”
 \footnote{\cite[pp.~39--40]{venecia}}














\subsection{Sacro Imperio}


\subsubsection{La Querella de las investiduras}

Durante este período del siglo XI, vivieron varios personajes
reconocidos de la Cristiandad, como el cid campeador, Guillermo
el conquistador de Inglaterra, Enrique IV de Alemania, además
de la primera cruzada, tema que trataré.

Cómo se pudo leer, está Enrique IV, nombrado en la Alexiada;
donde quisiera presentar la siguiente cita:

\textit{``[...] Había vuelto a enviar al rey de Alemania embajadores
que presidía el llamado Metimnes y por carta instaba con energía a no
retrasar más la ayuda y a invadir con gran rapidez Longibardía al frente
de sus tropas y contigentes extranjeros y poder expulsarlo del Ilírico.
A continuación le comunicó al rey de Alemania que, si obraba de este
modo, le quedaría muy reconocido y le aseguraba que la boda prometida
sería ultimada a través de los embajadores enviados por él.''}
\footnote{\cite[p.~215]{alexiadaV}}

En primer lugar, la autora hace referencia a Italia como ``Longibardía'',
en segundo lugar, la mención de Enrique IV como rey de Alemania, en
lugar de emperador. A la vez de referirse a si mismos cómo
``imperio de los romanos'' (en parte al ser de los romanos, y no
de Roma, ya que no la tenían...). Por lo cual, se puede ver
que se consideraban ellos como ``El imperio'' y no admitían que
Alemania lo fuese, ni siquiera como homólogo occidental.

Pues, en esa carta, el emperador Alejo I insta a Enrique IV a invadir
Italia, para distraer al rey de Sicilia, Roberto Guiscardo,
coincidiendo con la querella de las investiduras, donde Enrique IV
depone al Papa, cómo desarrollaré a continuación, siguiendo el libro de
``Ciudad cristiana'' de Rubén Calderon Bouchet, ya que a su vez
el Papa estaba aliado a Roberto Guiscardo, y fue él quien le coronó
rey de Sicilia.

\subsubsection{Desde otro punto de vista}

% Segunda revisión.

En la siguiente cita de Rubén Calderon Bouchet se desarrolló
desde el otro lado. El rey de Alemania, Enrique IV
se dirigió a Roma a deponer al Papa y poner su antipapa.
A su vez, el Papa Gregorio llamó a los normandos,
%que es Roberto Guiscardo, rey de Sicilia por el mismo papa Gregorio,
% -> cambio de renglón
o sea a Roberto Guiscardo quien habia sido coronado rey de Sicilia
por el mismo Papa Gregorio.
Guiscardo tuvo que dejar en manos de si hijo, Bohemundo de Tarento
(el mismo que sería príncipe de Antioquia en la primer cruzada)
en Grecia, mientras él respondía al llamado del Papa
contra Enrique IV.

Mientras que el emperador Alejo I del imperio Bizantino le había mandado
una carta, con una oferta de boda con una princesa bizantina,
si Enrique IV atacaba a los normandos.

¿Por qué atacó el emperador Enrique IV, por deponer al Papa Gregorio,
por la alianza con Alejo I, o ambos?

¡Una escena digna del príncipe de Maquiavelo!

\textit{
	``Enrique IV nuevamente dueño de la situación en Alemania por
	un oportuno triunfo militar convocó a un concilio a todos
	los obispos adictos y volvió a deponer al papa Gregorio
	y designó en su lugar a Wilberto de Ravena. Sin perder
	el tiempo marchó sobre Roma y colocó a su antipapa al frente
	de la cristiandad con el nombre de Clemente III.
	En la capital del cristianismo recibió de manos de Clemente
	la corona imperial.
}

\textit{
	``Gregorio no reconoció la autoridad de Enrique y luego
	de haber lanzado sobre él una nueva excomunión sin éxito
	se refugió en el castillo de Sant'Angelo y desde allí
	pidió ayuda militar a los normandos.
}

% -> ver en la fuente.
\textit{
	``Estos aventureros apenas desbastados de sus antecedentes
	bárbaros eran un arma de doble filo. Combatieron contra
	Enrique IV, tomaron la ciudad de Roma y, después de librar
	al Papa de su prisión en Sant'Angelo, sometieron la ciudad
	pontificia a un prolijo saqueo. Con este acto de bandolerismo
	se enajeraron la adhesión de los romanos que prefirieron
	en adelante la protección del excomulgado y su antipapa.
}

\textit{
	``Gregorio buscó refugio en la abadía de Montecassino y luego
	se dirigió a Salerno donde renovó la excomunión contra
	Enrique y Clemente y dirigió a todos los fieles una carta
	encíclica para exponerles la situación de la Iglesia.
}

\textit{
	``El asunto afectaba a toda la cristiandad en sus dos cabezas.
	En 1085 se reunióun concilio para examinar el problema
	de las excomuniones. Una y otra potestad tuvo en la
	asamblea sus defensores y sus impugnadores y la prolongación
	de la disputa reveló su completa esterilidad. El 25 de mayo de
	ese año murió Gregorio VII en Salerno.''
}\footnote{\cite[p.~871--872]{ciudad}}

\subsubsection{Relación entre ambos imperios}

Continuemos viendo la relación entre el sacro imperio y
el imperio bizantino. Ambos se consideraban imperio romano.
Los emperadores alemanes decían que no pueden llamarse romano
si no tienen Roma, pero Constantinopla se decía ``de los
romanos''.

\textit{``El hijo de Luis el Piadoso, también de nombre Luis,
	pero conocido como el Germánico, en una carta al
	emperador de Bizancio Basilio I, le explica el uso del
	título de Emperador de los Romanos que parece sorprender
	no poco al bizantino.}

\textit{``Conviene que sepas que si no fuéramos emperador de los
	romanos tampoco lo seríamos de los francos. Hemos recibido
	ese nombre y ese título de los mismos romanos, porque
	entre ellos brilló con viva claridad la cima de tan gran
	sublimidad y de tan prestigiosa apelación.
	La decisión de Dios nos hizo asumir el gobierno del pueblo
	y de la ciudad, así como la defensa y exaltación de la madre
	de todas las Iglesias, y es ella quien ha conferido a los
	primeros príncipes de nuestra dinastía primero la autoridad
	real y luego la imperial...
	Es por la unción que le confió el sumo pontífice que
	nuestro antepasado Carlos el Grande, primero de nuestra
	nación y de nuestra familia, ha sido declarado emperador
	y se ha convertido en el Cristo del Señor en razón
	de su piedad inmensa''.''}
\footnote{\cite[p.~846]{ciudad}}

Como cuenta y cita Bouchet, el conflicto entre ambos imperios
sobre referirse a si mismos como ``imperio de los romanos''.
A pesar de que en ese caso es entre Luis el gérmanico y
Basilio I, la princesa bizantina Ana Comnena, en su libro
la Alexiada, menciona al Sacro Imperio como reino de Alemania,
y al imperio de Constantinopla como Imperio de los Romanos.

\textit{
	````Así como en razón de nuestra fe en Cristo pertenecemos
	a la raza de Abraham... así hemos recibido el gobierno del
	Imperio Romano, en razón de nuestra buena ortodoxia.
	Los griegos, por el contrario, por sus opiniones heréticas,
	han dejado de ser emperadores de los romanos''.}

% -> Última de la segunda revisión
\textit{
	``Folz une los elementos esenciales de este documento
	y determina la idea carolingia del imperio
	en cinco puntos:\\
	a. Paridad del Imperio Romano y el Bizantino. Sin
	tendencia universalista.\\
	b. Se niega al emperador bizantino el título de emperador
	de los romanos porque no corresponde a una realidad política.\\
	c. Sublimidad del imperio, porque su dignidad reside cerca de
	Dios en el fuste de la piedad.\\
	d. La fuente de legitimación del imperio está en Roma.
	e. Los francos son portadores del título imperial por su
	ortodoxia y este imperio de substancia germánica es una a
	pesar de su división en reinos.''''}
\footnote{\cite[p.~846-847]{ciudad}}

Con esto se puede ver el conflicto de ambos imperios por
quien tenía que llamarse de los romanos. Siendo ambos los
que se lo adjudicaban.




















\subsection{El Danubio}
\input{politica/danuvio}

\subsection{Primera Cruzada}
\input{politica/cruzada}


\newpage

\section{Conclusión}
\input{contexto/conclusion}
\newpage

\section{Bibliografía}
\bibliography{contexto/biblio}
\newpage

\setstretch{1.5}
\tableofcontents

\end{document}

