
	El contexto científico en el que se va a encontrar el trabajo, se trata de recopilar información de los diversos estudios que hablen del tema elegido. Ya que el tema, no fue tocado por nadie de forma directa, los antecedentes del estudio son los que investigaron brindando la información necesaria para la reconstrucción de cómo pudo ser la vida en la Constantinopla del siglo XI, y los novelistas, que estudiando y leyendo para la construcción histórica del escenario de sus novelas, estudiaron el tema de forma directa, como es el objetivo del trabajo.

Interrogantes e hipótesis del trabajo:

¿Cómo se vivía en la Constantinopla del siglo XI?

¿Cuál era el plan de estudios?

¿Cómo era la salud pública?

¿Cómo se organizaba el ejército?

¿Quiénes eran los personajes político-militares-culturales?

	A pesar de que varias de las anteriores interrogantes ya las habrán tratado de forma directa, en este trabajo se recopilaría dicha información con el fin de reconstruir la vida de Constantinopla del siglo XI, como figura anteriormente.

	Por ejemplo, en el libro XV de “La Alexiada” se relata un gran hospital público, y un colegio, publico, para huérfanos, junto con una nueva didáctica de dicho colegio. Además, se relata una nueva formación militar.

		Y, en el capítulo IV de “Filosofía Bizantina” se describe el plan de estudios que había diseñado el catedrático de la universidad de Constantinopla, Miguel Psellos.

	El método que se va a usar en el trabajo, es el método testimonial. Ya que se va a confiar en los testimonios de los que estudiaron el tema, y en la fuente primaria de “La Alexiada”.
