

\subsubsection{El orfanato}

% Introducción, orfanato-escuela
\textit{``
	Dedicó el día siguiente entero al cuidado de
	cautivos y recién llegados. Repartió entre todos
	aquellos de sus allegados, que sabía que llevaban una
	vida honesta, y entre los higúmenos [abades] de los sagrados
	monasterios a todos los niños que habían quedados
	privados de padres y estaban sumidos en la amarga desgracia
	de la orfandad, y les recomendó que no los criasen como
	esclavos, sino como seres libres, considerándolos merecederos
	de una completa formación e intruyéndolos en las
	Sagradas Escrituras. También entregó algunos al orfanato que
	él había fundado y que estaba pensado más como escuela para
	quienes quisieran aprender, a fin de que sus directores
	les enseñeran el ciclo completo de estudios [trivium y quadrivium].
	''}
\footnote{\cite[pp.~605--606]{alexiadaXV}}

Para concluir el capítulo sobre educación, presento la narración sobre
el orfanato-escuela fundado por el emperador según la autora.
Dónde describe brevemente el lugar, menciona quienes van, el método
didáctico que se empezó a usar ahí.

% Descripción, educación en el orfanato
\textit{``
	[...]. A la derecha del gran templo [de los apóstoles] hay una escuela
	primaria para los niños huérfanos procedentes de toda variedad de razas,
	en donde un maestro imparte la clase y los niños se colocan en torno
	a él, unos atemorizados por las preguntas sobre grámatica, otros
	escribiendo la denominada esquedografía. Allí es posible ver a un
	latino que se está instruyendo, a un escita que aprende griego, a un
	romano manejando textos griegos y a un griego iletrado que aprende hablar
	correctamente griego. Esos eran los afanes de Alejo sobre la formación
	intelectual. En cuanto a la técnica de la esquedografía, diremos que es
	un invento de los más recientes y originario de nuestra generación. [...]
	ahora el estudio de estos maestros, de los poetas, de los historiadores
	y de sus cualidades no ocupa siquiera un lugar secundario. El único interés
	es el juego, los demás trabajos están prohibidos.
	''}
\footnote{\cite[p.~609]{alexiadaXV}}

Por lo que se puede ver que no se necesitaba de mucho dinero para acceder
a una educación completa (trivium y quadrivium), además de poder leer
sobre el método didáctico que se estaba poniendo a prueba en dicho momento.





















